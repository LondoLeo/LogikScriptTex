\documentclass[12pt,a4paper]{report}
\usepackage[utf8]{inputenc}
\usepackage[T1]{fontenc}
\usepackage{amsmath, amssymb, amsfonts}
\usepackage{stmaryrd}
\usepackage[ngerman]{babel}

\usepackage{bussproofs} % Deduktionsbäume
\usepackage{sfmath}

\newcommand{\abs}[1]{\left\lvert #1 \right\rvert}
\newcommand{\ninf}{\sum_{n=0}^\infty}
\newcommand{\ksum}[2]{\sum_{k=#1}^#2}
\newcommand{\limes}[2]{\lim_{#1 \rightarrow #2}}
\newcommand{\ra}{\rightarrow}
\newcommand{\RA}{\Rightarrow}
\newcommand{\LR}{\Leftrightarrow}
\newcommand{\integral}[1]{\int #1 \,\mathrm{d}x}
\newcommand{\Integral}[3]{\int_{ #1 }^{ #2 } #3 \,\mathrm{d}x}
\newcommand{\real}{\mathbb{R}}
\newcommand{\mat}[1]{\begin{pmatrix}#1 \end{pmatrix}}
\newcommand{\ls}{\newline\newline}
\newcommand{\diff}[2]{\frac{ \partial #1 }{ \partial #2 } }
\newcommand{\intervall}{\mathcal{I}}
\newcommand{\logicand}{\wedge}
\newcommand{\logicor}{\vee}
\title{Zusammenfassung - Logik und Logikprogrammierung}
\date{\today}
\author{Marcus Faulstich}

\newcommand{\define}[1]{\section{\blue{Definition #1}}}
\newcommand{\satz}[1]{\section{\green{Satz #1}}}
\newcommand{\lemm}[1]{\section{\green{Lemma #1}}}

\usepackage{hyperref}                   % hyperlinks
\hypersetup{
    colorlinks=true,
    linktoc=all,
    linkcolor=black
}

\usepackage{fancyhdr}                   % Inhaltsverzeichnis
\pagestyle{fancy}                       % im Footer
\renewcommand{\headrulewidth}{0pt}
\fancypagestyle{plain}{
    \fancyhf{}
    \cfoot{\hyperlink{contents}{\textbf{Inhaltsverzeichnis}}}
}
\fancyhf{}
\cfoot{\hyperlink{contents}{\textbf{Inhaltsverzeichnis}}}


\renewcommand{\familydefault}{\sfdefault}   % sans serif font

% Highlighting Macros
\newcommand{\red}[1]{\textcolor[rgb]{0.9,0.2,0.2}{#1}}
\newcommand{\green}[1]{\textcolor[rgb]{0.1,0.6,0.1}{#1}}
\newcommand{\blue}[1]{\textcolor[rgb]{0.2,0.2,1}{#1}}

\begin{document}
\maketitle
\hypertarget{contents}{}
\tableofcontents

\chapter{Aussagenlogik}

\define{Atomare Formeln}
Eine \red{atomare Formel} hat die Form $ p_i $. Diese Formeln werden durch folgenden Induktiven Prozess definiert:
\begin{itemize}
    \item Alle atomaren Formeln und $ \perp $ sind Formeln
    \item Falls $ \varphi $ und $ \psi  $ Formeln sind, sind auch $ (\varphi \logicand \psi),~ (\varphi \logicor \psi),~ (\varphi \ra \psi) $ und $ \lnot \varphi $ Formeln
    \item Nichts ist Formel, was sich nicht mittels dieser Regeln erzeugen lässt
\end{itemize}
\textbf{Bezeichungen: }
\begin{itemize}
    \item Falsum: $ \perp $
    \item Konjunktion: $ \logicand $
    \item Disjunktion: $ \logicor $
    \item Einseitige Implikation: $ \ra $
    \item Gegenseitige Implikation: $ \leftrightarrow $
    \item Negation: $ \lnot $
\end{itemize}
\textbf{Präzedenz der Operatoren:}
\begin{align*}
    \leftrightarrow &~ \textnormal{bindet am stärksten}\\
    \ra & \dots\\\logicor & \dots\\\logicand & \dots\\
    \lnot &~ \textnormal{bindet am stärksten}
\end{align*}

\section{\blue{Natürliches Schließen}}
Ein mathematischer Beweis zeigt, wie die Behauptung aus den Voraussetzungen folgt. Analog zeigt ein \textbf{Beweisbaum} (\textit{Herleitung, Deduktion}), wie eine Formel aus anderen Formeln folgt.
\ls
Diese Deduktionen sind Bäume, deren Knoten mit Formeln beschriftet sind.
\begin{itemize}
    \item an den Blättern stehen Voraussetzungen (\red{Hypothesen})
    \item an den inneren Knoten stehen Teilergebnisse und Begründungen
    \item an der Wurzel steht die Behauptung (\red{Konklusion})
\end{itemize}
\section{\blue{Konstruktion von Deduktionen}}
\subsection{\green{Konjuktionseinführung/-elimination}}
\begin{prooftree}
\AxiomC{$ \varphi $}
\AxiomC{$ \psi $}
\BinaryInfC{$ \varphi \logicand \psi $}
\end{prooftree}
\begin{center}
    \begin{minipage}{0.2\linewidth}
        \begin{prooftree}
            \AxiomC{$ \varphi \logicand \psi $}
            \UnaryInfC{$ \varphi $}
        \end{prooftree}
    \end{minipage}
    \begin{minipage}{0.2\linewidth}
        \begin{prooftree}
            \AxiomC{$ \varphi \logicand \psi $}
            \UnaryInfC{$ \psi $}
        \end{prooftree}
    \end{minipage}
\end{center}
\subsection{\green{Implikationseinführung/-elimination}}
\begin{prooftree}
    \AxiomC{$ [\varphi] $}
    \noLine
    \UnaryInfC{$ \vdots $}
    \noLine
    \UnaryInfC{$ \psi $}
    \UnaryInfC{$ \varphi \ra \psi $}
\end{prooftree}

\begin{prooftree}
    \AxiomC{$ \varphi $}
    \AxiomC{$ \varphi \ra \psi $}
        \BinaryInfC{$ \psi $}
\end{prooftree}
\subsection{\green{Disjunktionseinführung/-elimination}}
\begin{center}
    \begin{minipage}{0.2\linewidth}
        \begin{prooftree}
            \AxiomC{$ \varphi $}
            \UnaryInfC{$ \varphi \logicor \psi $}
        \end{prooftree}
    \end{minipage}
    \begin{minipage}{0.2\linewidth}
        \begin{prooftree}
            \AxiomC{$ \psi $}
            \UnaryInfC{$ \varphi \logicor \psi $}
        \end{prooftree}
    \end{minipage}
\end{center}
\begin{prooftree}
    \AxiomC{$ \varphi \logicor \psi $}
    \AxiomC{$ [\varphi] $}
        \noLine
        \UnaryInfC{$ \vdots $}
        \noLine
        \UnaryInfC{$ \sigma $}
    \AxiomC{$ [\psi] $}
        \noLine
        \UnaryInfC{$ \vdots $}
        \noLine
        \UnaryInfC{$ \sigma $}
    \TrinaryInfC{$ \sigma $}
\end{prooftree}
\subsection{\green{Negationseinführung/-elimination}}
\begin{prooftree}
    \AxiomC{$ [\varphi] $}
    \noLine
    \UnaryInfC{$ \vdots $}
    \noLine
    \UnaryInfC{$ \perp $}
    \UnaryInfC{$ \lnot \varphi $}
\end{prooftree}
\begin{prooftree}
    \AxiomC{$ \lnot \varphi $}
    \AxiomC{$ \varphi $}
    \BinaryInfC{$ \perp $}
\end{prooftree}
\subsection{\green{Reductio ad absurdum}}
\begin{prooftree}
    \AxiomC{$ [\lnot \varphi] $}
    \noLine
    \UnaryInfC{$ \vdots $}
    \noLine
    \UnaryInfC{$ \perp $}
    \UnaryInfC{$ \varphi $}
\end{prooftree}

\define{Syntaktische Folgerung, Theorem}
Für eine Formelmenge $ \Gamma $ und eine Formel $ \varphi $ schreiben wir
\[
    \Gamma ~\red \vdash ~\varphi
\]
wenn es eine Deduktion gibt mit Hypothesen aus $ \Gamma $ und Konklusion $ \varphi $. Wir sagen "$ \varphi $ ist eine \red{syntaktische Folgerung} von $ \Gamma $"\ls
Eine Formel $ \varphi $ ist ein \red{Theorem}, wenn $ \emptyset ~ \vdash ~ \varphi $ gilt.
\satz{}
Für alle Formeln $ \varphi $ und $ \psi $ gelten
\[
    \{\lnot (\varphi \logicor \psi)\} ~ \vdash ~ \lnot \varphi \logicand \lnot \psi
\]
\[
    \{\lnot \varphi \logicand \lnot \psi \}~ \vdash ~ (\lnot \varphi \vee \psi)
\]
\[
    \{\lnot \varphi \vee \lnot \psi\} ~ \vdash ~ \lnot(\varphi \wedge \psi)
\]
\[
    \{\lnot (\varphi \wedge \psi) ~\vdash~ \lnot \varphi \vee \lnot \psi\}
\]
\define{Wahrheitswertemengen}
\begin{itemize}
    \item \red{Boolsche Logik} $ B = \{0,1\} $
    \item \red{Kleene-Logik} $ K_3 = {0,\frac{ 1 }{ 2 } , 1} $
    \item \red{Fuzzy-Logik} $ F = [0,1] $
    \item \red{unendl. Boolsche Logik} $ B_\real =  $ Menge der Teilmengen von $ \real $
    \item \red{Heyting-Algebra} $ H_\real =$ Menge der offenen Teilmengen von $ \real $
\end{itemize}
\define{W-Belegung}
Sei $ W $ eine Menge von Wahrheitswerten. Eine \red{$W $-Belegung} ist eine Abbildung $ \mathcal{B}:~ V \ra W $, wobei $ V \subseteq \{p_0, p_1,\dots\} $ eine Menge atomarer Formeln ist.\ls
Die $ W $-Belegung $ \mathcal{B}:~ V \ra W $ \red{\textit{passt}} zur Formel $ \varphi $, falls alle atomaren Formeln aus $ \varphi $ zu $ V $ gehören.\ls
Sei $ W $ ein Wahrheitswertebereich und $\mathcal{B}$ eine $ W $-Belegung. Induktiv über den Formelaufbau definieren wir den Wahrheitswert $ \red{\hat{\mathcal{B}}(\varphi)} \in W  $ jeder zu $ \mathcal{B} $ passenden Formel $ \varphi  $:
\begin{center}
    \begin{align*}
        \hat{\mathcal{B}}(\perp) &= 0_W\\[1em]
        \hat{\mathcal{B}}(p) &= \mathcal{B}(p) \qquad \textnormal{falls $ p $ atomare Formel}\\[1em]
        \hat{\mathcal{B}}(\varphi \wedge \psi) &= \hat{\mathcal{B}}(\varphi) \wedge_W \hat{\mathcal{B}}(\psi)\\[1em]
        \hat{\mathcal{B}}(\varphi \vee\psi) &= \hat{\mathcal{B}}(\varphi) \vee_W \hat{\mathcal{B}}(\psi)\\[1em]
        \hat{\mathcal{B}}(\varphi\ra \psi) &= \ra_W(\hat{\mathcal{B}}(\varphi),\hat{\mathcal{B}}(\psi))\\[1em]
        \hat{\mathcal{B}}(\lnot\varphi) & = \lnot_W(\hat{\mathcal{B}}(\varphi))
    \end{align*}
\end{center}
Im folgenden schreiben wir anstatt $ \hat{\mathcal{B}}(\varphi) $ nur $ \mathcal{B}(\varphi) $

\define{Abbildungen von Wahrheitswerten}
Sei $ W $ eine Menge und $ R \subseteq W \times W $ eine binäre Relation.
\begin{itemize}
    \item $ R $ ist \red{reflexiv}, wenn $ (a,a) \in R $ für alle $ a \in W $ gilt
    \item $ R $ ist \red{antisymmetrisch}, wenn $ (a,b),(b,a) \in R $ impliziert, dass $ a = b $ gilt
    \item $ R $ ist \red{transitiv}, wenn $ (a,b),(b,a) \in R $ impliziert, dass $ (a,c) \in R $ gilt
    \item $ R $ ist eine \red{Ordnungsrelation}, wenn $ R $ reflexiv, antisymmetrisch und transitiv ist. In diesem Fall heißt das Paar $ (W,R) $ eine \red{partiell geordnete Menge}
\end{itemize}
\define{Schranken von Wahrheitswertemengen}
Sei $ (W, \leq) $ partiell geordnete Menge, $ M \subseteq W $ und $ a \in W $.
\begin{itemize}
    \item $ a $ ist \red{obere Schranke von $ M $}, wenn $ m \leq a $ für alle $ m \in M $ gilt
    \item $ a $ ist \red{kleinste obere Schranke} oder \red{Supremum von $ M $}, wenn $ a $ obere Schranke von $ M $ ist und wenn $ a \leq b $ für alle oberen Schranken $ b $ von $ M $ gilt
    \item $ a $ ist \red{untere Schranke von $ M $}, wenn $ a \leq m $ für alle $ m \in M $ gilt
    \item $ a $ ist \red{größte untere Schranke} oder \red{Infimum von M}, wenn $ a $ untere Schranke von $ M $ ist und wenn $ b \leq a  $ für alle unteren Schranken $ b $ von $ M $ gilt.
\end{itemize}
\define{Verband}
Ein \red{(vollständiger) Verband} ist eine partiell geordnete Menge $ (W, \leq) $, in der jede Menge $ M \subseteq W $ ein Supremum und ein Infimum hat.\ls
In einem Verband $ (W,\leq) $ definieren wir:
\begin{itemize}
    \item \red{$ 0_W $}$ = \inf W $ und $ \red{1_W} = \sup W $
    \item $ a \red{\wedge_W} b = \inf{a,b} $ und $ a \red{\vee_W} b = \sup{a,b} $ für $ a,b \in W $
\end{itemize}
\define{Wahrheitswertebereiche}
Ein \red{Wahrheitswertebereiche} ist ein Tupel $ (W, \leq, \ra_W, \lnot_W) $, wobei $ (W, \leq) $ ein Verband und $ \ra_W:~ W^2 \ra W $ und $ \lnot_W:~ W \ra W $ Funktionen sind

\begin{itemize}
    \item \red{Boolscher Wahrheitswertebereich $ B $}
    \begin{align*}
        \textnormal{Grundmenge: } & B = {0,1}
        &\textnormal{Ordnung:}  \leq\\
        & \red{\lnot_B}(a) = 1 - a & \red{\ra_B}(a,b) = \max(b, 1-a)\\
        & \red{0_B} = 0 \qquad \red{1_B} = 1\\
        & a \red{\wedge_B} b = \min(a,b) & a \red{\vee_B} b = \max(a,b)
    \end{align*}
    \item \red{Kleenescher Wahrheitswertebereich $ K_3 $}
    \begin{align*}
        \textnormal{Grundmenge: } & K_3 = \{0, \frac{ 1 }{ 2 } , 1\} & \textnormal{Ordnung: } \leq\\
        & \red{\lnot_{K_3}}(a) = 1 - a & \red{\ra_{K_3}}(a,b) = \max(b,1-a)\\
        & \red{0_{K_3}} = 0 \qquad \red{1_{K_3}} = 1\\
        & a \red{\wedge_{K_3}} b = min(a,b) & a \red{\vee_{K_3}} b = \max(a,b)
    \end{align*}
    \item \red{Fuzzy-Wahrheitswertebereich}
    \begin{align*}
        \textnormal{Grundmenge: } & F = [0,1] & \textnormal{Ordnung: } \leq\\
        &\red{\lnot_F}(a) = 1-a & \red{\ra_F} (a,b) = \max(b,1-a)\\
        &\red{0_F} = 0 \qquad \red{1_F} = 1\\
        &a \red{\wedge_F} b = \min(a,b) & a \red{\vee_F} b = \max(a,b)
    \end{align*}
    \item \red{unendl. Boolscher Wahrheitswertebereich}
    \begin{align*}
        \textnormal{Grundmenge: } & B_\real = \{A \mid A \subseteq \real\} & \textnormal{Ordnung: } \subseteq\\
        &\red{\lnot_{B_\real}}(A) = \real \backslash A & \red{\ra_{B_\real}} (A,B) = B \cup \real \backslash A\\
        &\red{0_{B_\real}} = \emptyset \qquad \red{1_{B_\real}} = \real\\
        &A \red{\wedge_{B_\real}} B = A \cap B & A \red{\vee_{B_\real}} B = A \cup B
    \end{align*}
    \item \red{Heytingscher Wahrheitswertebereich}
    \begin{align*}
        \textnormal{Grundmenge: } & H_\real = \{A \subseteq \real \mid A \textnormal{ ist offen}\} & \textnormal{Ordnung: } \subseteq\\
        &\red{\lnot_{H_\real}}(A) = \textnormal{Inneres}(\real \backslash A) & \red{\ra_{H_\real}} (A,B) = \textnormal{Inneres}(B \cup \real \backslash A)\\
        &\red{0_{H_\real}} = \emptyset \qquad \red{1_{H_\real}} = \real\\
        &A \red{\wedge_{H_\real}} B = A \cap B  & A \red{\vee_{H_\real}} B = A \cup B
    \end{align*}
\end{itemize}

\define{Semantische Folgerung, Tautologie}
Sei $ W $ ein Wahrheitswertebereich.\newline
Eine Formel $ \varphi $ heißt eine \red{$ W $-Folgerung} der Formelmenge $ \Gamma $, falls für jede $ W $-Belegung $ \mathcal{B} $, die zu allen Formeln aus $ \Gamma \cup \{\varphi\} $ passt, gilt:
\[
    \inf\{\mathcal{B}(\gamma) \mid \gamma \in \Gamma\} \leq \mathcal{B}(\varphi)
\]
Wir schreiben \red{$ \Gamma \vDash_W \varphi $}, falls $ \varphi $ eine $ W $-Folgerung von $ \Gamma $ ist.\ls
Eine \red{$ W$-Tautologie}  ist eine Formel $ \varphi $ mit $ \emptyset \vDash_W \varphi $, d.h. $ \mathcal{B}(\varphi) = 1_W $ für alle passenden $ W$-Belegungen.

\subsection{\green{Korrektheitslemma für natürliches Schließen in $B$}}
Sei $ D $ eine Deduktion mit Hypothesen in der Menge $ \Gamma $ und Konklusion $ \varphi $. Dann gilt
\[
    \Gamma \vDash_B \varphi \quad \textnormal{ d.h. } \quad \inf\{\mathcal{B}(\gamma) \mid \gamma \in \Gamma\} \leq \mathcal{B}(\gamma)
\]
für alle passenden $ B $-Belegungen $ \mathcal{B} $.
\subsection{\green{Korrektheitssatz für natürliches Schließen in $B$}}
Für jede Menge von Formeln $ \Gamma $ und jede Formel $ \varphi $ gilt
\[
    \Gamma \vdash \varphi ~\RA~ \Gamma \vDash_B \varphi
\]
\paragraph{Korollar} Jedes Theorem ist eine $ B $-Tautologie.

\subsection{\green{Korrektheitssatz für natürliches Schließen in $ B_\real $}}
Für jede Menge von Formeln $ \Gamma $ und jede Formel $ \varphi $ gilt
\[
    \Gamma \vdash \varphi ~\RA~ \Gamma \vDash_{B_\real} \varphi
\]
\paragraph{Korollar} Jedes Theorem ist eine $ B_\real $-Tautologie.

\subsection{\green{Korrektheitssatz für natürliches Schließen in $ H_\real $}}
Für jede Menge von Formeln $ \Gamma $ und jede Formel $ \varphi $ gilt
\[
    \Gamma \vdash \varphi \textnormal{ ohne (raa)} ~\RA~ \Gamma \vDash_{H_\real} \varphi
\]
\paragraph{Korollar} Jedes (raa)-freie Theorem ist eine $ H_\real $-Tautologie.

\define{Konsistente Mengen}
Sei $ \Gamma $ eine Menge von Formeln. $ \Gamma $ heißt \red{inkonsistent}, wenn $ \Gamma \vdash \perp $ gilt. Sonst heißt $ \Gamma $ \red{konsistent}.
\subsection{\green{Lemma}}
Sei $ \Gamma $ eine Menge von Formeln und $ \varphi $ eine Formel. Dann gilt
\[
    \Gamma \nvdash \varphi \Longleftrightarrow \Gamma \cup \{\lnot \varphi\} \textnormal{ konsistent}
\]
\subsection{\blue{Maximal Konsistente Mengen}}
Eine Formelmenge $ \Delta $ ist \red{maximal konsistent}, wenn sie konsistent ist und wenn gilt
\[
    \Sigma \supseteq \Delta \textnormal{ konsistent} ~\RA~ \Sigma = \Delta
\]
\subsection{\green{Satz}}
Jede konsistente Formelmenge $ \Gamma $ ist in einer maximal konsistenten Formelmenge $ \Delta $ enthalten.
\subsection{\green{Lemma 1}}
Sei $ \Delta $ maximal konsistent und gelte $ \Delta \vdash \varphi $. Dann gilt $ \varphi \in \Delta $.
\subsection{\green{Lemma 2}}
Sei $ \Delta $ maximal konsistent und $ \varphi $ Formel. Dann gilt
\[
    \varphi \notin \Delta ~\Longleftrightarrow~ \lnot \varphi \in \Delta
\]
\define{Erfüllbare Mengen}
Sei $ \Gamma $ eine Menge von Formeln.\newline
$ \Gamma $ heißt \red{erfüllbar}, wenn es passende $ B $-Belegungen $ \mathcal{B} $ gibt mit $ \mathcal{B}(\gamma) = 1_B $ für alle $ \gamma \in \Gamma $.
\subsection{\green{Satz}}
Sei $ \Delta $ eine maximal konsistente Menge von Formeln. Dann ist $ \Delta $ erfüllbar.
\subsection{\green{Lemma}}
Sei $ \Gamma $ eine Menge von Formeln und $ \varphi $ eine Formel. Dann gilt
\[
    \Gamma \nvDash_B \varphi ~\Longleftrightarrow~ \Gamma \cup \{\lnot \varphi\} \textnormal{ erfüllbar}
\]
\subsection{\green{Vollständigkeitssatz}}
Sei $ \Gamma $ eine Menge von Formeln, $ \varphi $ eine Formel und $ W $ einer der Wahrheitswertebereiche $ B, K_3, F, B_\real, H_\real $. Dann gilt
\[
    \Gamma \vDash_W \varphi ~\RA~ \Gamma \vdash \varphi
\]
Insbesondere ist jede $ W $-Tautologie ein Theorem.

\section{\blue{Vollständigkeit und Korrektheit}}
\subsection{\green{Satz}}
Seien $ \Gamma $ eine Menge von Formeln und $ \varphi $ eine Formel. Dann gilt
\[
    \Gamma \vdash \varphi ~\Leftrightarrow~ \Gamma \vDash_B \varphi
\]
Insbesondere ist eine Formel genau dann eine $ B $-Tautologie, wenn sie ein Theorem ist.
\paragraph{Bemerkung} Gilt für jede Boolsche Algebra, also auch $ B_\real $, und für Heyting ohne (raa).
\subsection{\blue{Definition Äquivalent}}
Zwei Formeln $ \alpha, \beta $ heißen \red{äquivlent} ($ \alpha \red{\equiv} \beta $), wenn für alle passenden $ B $-Belegungen $ \mathcal{B} $ gilt:
\[
    \mathcal{B}(\alpha) = \mathcal{\beta}
\]
\subsection{\green{Zusammenhänge}}
Seien $ \alpha, \beta $ zwei Formeln, dann gilt:
\[
    \alpha \equiv \beta ~\Leftrightarrow~ (\alpha \leftrightarrow \beta) \textnormal{ ist Theorem}
\]
\[
    \alpha \textnormal{ ist Theorem} ~\Leftrightarrow~ \alpha \equiv \lnot\perp
\]
\subsection{\green{Kompaktheitssatz}}
Sei $ \Gamma $ eine unter Umständen unendliche Menge von Formeln. Dann gilt
\[
    \Gamma \textnormal{ unerfüllbar} ~\Leftrightarrow~ \exists \Gamma' \subseteq \Gamma \textnormal{ endlich}:~ \Gamma' \textnormal{ unerfüllbar}
\]
\section{\blue{Erfüllbarkeitsproblem}}
\begin{center}
    \begin{align*}
        &\textnormal{Eingabe: Formel } \gamma\\
        &\textnormal{Frage: existiert eine $ B $-Belegung $ \mathcal{B} $ mit $ \mathcal{B}(\gamma) = 1_B$}
    \end{align*}
\end{center}
Offensichtliche Lösung: probiere alle Belegungen durch (Wahrheitswertetabelle), jedoch exponentielle Zeit. Ziel ist also ein spezieller Algorithmus für syntaktisch eingeschränkte Formeln.
\subsection{\blue{Definition Hornformeln}}
Eine \red{Hornklausel} hat die Form
\[
    (\lnot \perp \wedge p_1 \wedge p_2 \wedge \dots \wedge p_n) \ra q
\]
für $ n \geq 0 $, atomare Formeln $ p_1,\dots,p_n $ und $ q $ atomare Formel oder $ q = \perp $.\ls
Eine \red{Hornformel} ist eine Konjuktion von Hornklauseln der Form
\[
    \bigwedge_{1 \leq i \leq n} (M_i \ra q_i) \textnormal{ oder } \{(M_i \ra q_i) \mid 1 \leq i \leq n\}
\]
\subsection{\red{Markierungsalgorithmus}}
\textbf{Eingabe:} endliche Menge $ \Gamma $ von Hornklauseln\newline
\textbf{Ausgabe:} \{erfüllbar, unerfüllbar\}\newline
\textbf{Algorithmus:}
\begin{itemize}
    \item[\blue{(1)}] Solange es eine Hornklausel $ M \ra q $ in $ \Gamma $ gibt, so dass alle $ p \in M $ markiert sind (oder $ M = \emptyset $) und $ q $ unmarkierte atomare Formel ist, dann markiere $ q $ in allen Hornklauseln
    \item[\blue{(2)}] Wenn $ \Gamma $ eine Hornklausel der Form $ M \ra \perp $ enthält, in der alle $ p \in M $ markiert sind
    \item[->] dann gebe ''überfüllbar'' aus
    \item[->] ansonsten gebe ''erfüllbar'' aus
\end{itemize}
\subsection{\blue{Definition SLD-Resolution}}
Sei $ \Gamma $ eine Menge von Hornklauseln. Eine \red{SLD-Resolution aus $ \Gamma $} ist eine Folge $ (M_0 \ra \perp, M_1 \ra \perp,\dots,M_m \ra \perp) $ von Hornklauseln mit
\begin{itemize}
    \item $ (M_0 \ra \perp) \in \Gamma $
    \item für alle $ 0 \leq n < m $ existiert $ (N\ra q) \in \Gamma$ mit $ q \in M_n $ und $ M_{n+1} = M_n\backslash\{q\}\cup N$
\end{itemize}

\chapter{Prädikatenlogik}
\section{\blue{Syntax der Prädikatenlogik}}
\subsection{\blue{Definition Signatur}}
Eine \red{Signatur} ist ein Tripel $ \Sigma = (\Omega, \texttt{Rel}, \texttt{ar})$, wobei $ \Omega $ und \texttt{Rel} disjunkte Mengen von \red{Funktions- und Relationsnamen} sind und $ \texttt{ar}:\Omega \cup \texttt{Rel} \ra \mathbb{N} $ eine Abbildung ist.
\subsection{\blue{Definition $ \Sigma $-Terme}}
Die \red{Menge der Variablen} ist $ \texttt{Var} = \{x_0,x_1,\dots\} $.\ls
Sei $ \Sigma $ eine Signatur. Die Menge $ \red{T_\Sigma} $ der \red{$ \Sigma $-Terme} ist induktiv definiert:
\begin{itemize}
    \item Jede Variable ist ein Term, d.h. $ \texttt{Var} \subseteq T_\Sigma $
    \item Ist $ f \in \Omega $ mit $ \texttt{ar}(f) =k$ und sind $ t_1,\dots,t_k \in T_\Sigma $, so gilt: $ f(t_1,\dots,t_k) \in T_\Sigma $
    \item Nichts ist $ \Sigma $-Term, was sich nicht mittel der obigen Regeln erzeugen lässt.
\end{itemize}
\subsection{\blue{Definition Atomare $ \Sigma $-Formeln}}
Sei $ \Sigma $ Signatur. Die \red{atomaren $ \Sigma $-Formeln} sind die Zeichenketten der Form
\begin{itemize}
    \item $ R(t_1,t_2,\dots,t_k) $ falls $ t_1,\dots,t_k \in T_\Sigma $ und $ R \in \texttt{Rel} $ mit $ \texttt{ar}(R) = k $ oder
    \item $ t_1 = t_2 $, falls $ t_1, t_2 \in T_\Sigma $ oder
    \item $ \perp $
\end{itemize}
\subsection{\blue{Induktive Definition $ \Sigma $-Formeln}}
Sei $ \Sigma $ Signatur. \red{$ \Sigma $-Formeln} werden durch folgenden induktiven Prozess definiert:
\begin{itemize}
    \item Alle atomaren $ \Sigma $-Formeln sind $ \Sigma $-Formeln
    \item Falls $ \varphi, \psi~ \Sigma$-Formeln, sind auch $ (\varphi \wedge \psi), (\varphi \vee \psi), (\varphi \ra \psi) $ $ \Sigma $-Formeln
    \item Falls $ \varphi $ eine $ \Sigma $-Formel ist, ist auch $ \lnot \varphi $ eine $ \Sigma $-Formel
    \item Falls $ \varphi $ eine $ \Sigma $-Formel und $ x \in \texttt{Var} $, so sind auch $ \forall x \varphi $ und $ \exists x \varphi $ $ \Sigma $-Formeln
    \item Nichts, was sich nicht mittels der obigen Regeln erzeugen lässt, ist $ \Sigma $-Formel
\end{itemize}
Ist der Kontext klar, so sprechen wir einfach von \red{Termen, atomaren Formeln bzw. Formeln}.
\subsection{\blue{Definition Freie Variablen}}
Sei $ \Sigma $ eine Signatur. Die Menge \red{$ FV(\varphi) $} der \red{freien Variablen} eine $ \Sigma $-Formel $ \varphi $ ist induktiv definiert:
\begin{itemize}
    \item Ist $ \varphi $ atomare $ \Sigma $-Formel, so ist $ FV(\varphi) $ die Menge der in $ \varphi $ vorkommenden Variablen
    \item $ FV(\varphi~\Box~ \psi) = FV(\varphi) \cup FV(\psi) $ für $ \Box \in \{\wedge, \vee, \ra\} $
    \item $ FV(\lnot \varphi) = FV(\varphi) $
    \item $ FV(\exists x \varphi) = FV(\forall x \varphi) = FV(\varphi)\setminus \{x\}  $
\end{itemize}
Eine Formel $ \varphi $ ist \red{geschlossen} oder ein \red{$ \Sigma $-Satz}, wenn $ FV(\varphi) = \emptyset $ gilt.

\subsection{\blue{Definition $ \Sigma $-Struktur}}
Sei $ \Sigma $ eine Signatur. Eine \red{$ \Sigma $-Struktur} ist ein Tupel $ \mathcal{A} = (U_\mathcal{A}, (f^\mathcal{A})_{f\in \Omega}, (R^\mathcal{A})_{R \in \texttt{Rel}}) $, wobei
\begin{itemize}
    \item $ U_\mathcal{A} $ ist eine nichtleere Menge, das \red{Universum}
    \item $ R^\mathcal{A} \subseteq U^{\texttt{ar}(R)}_\mathcal{A} $ eine Relation der Stelligkeit $ ar(R) $ für $ R \in \texttt{Rel} $
    \item $ f^\mathcal{A}: U^{\texttt{ar}(f)}_\mathcal{A} \ra U_\mathcal{A} $ eine Funktion der Stelligkeit $ \texttt{ar}(f) $ für $ f \in \Omega $
\end{itemize}

\subsection{\blue{Definition Variableninterpretation}}
Im Folgenden sei $ \Sigma $ Signatur, $ \mathcal{A} $ Struktur und $ \rho: \texttt{Var} \ra U_\mathcal{A} $ eine Abbildung \red{(Variableninterpretation)}.\ls
Die Abbildung \red{$ \rho': T_\Sigma \ra U_\mathcal{A} $} wird induktiv für $ t \in T_\Sigma $ definiert:
\begin{itemize}
    \item Ist $ t \in \texttt{Var} $, so setze $ \rho'(t) = \rho(t) $
    \item Ansonsten existieren $ f \in \Omega $ mit $ \texttt{ar}(f) = k \textnormal{ und } t_1,\dots,t_k \in T_\Sigma $ mit $ t = f(t_1,\dots,t_k) $. Dann setze $ \rho'(t) = f^\mathcal{A}(\rho'(t_1),\dots,\rho'(t_k)) $
\end{itemize}
Die Abbildung $ \rho' $ ist die übliche ''Auswertungsabbildung''. Zur Vereinfachung wird anstelle von $ \rho'(t) $ auch $ \rho(t) $ geschrieben.

\subsection{\green{Gültigkeit einer $ \Sigma $-Formel}}
Für eine Formel $ \varphi $ definieren wir die Gültigkeit in einer Struktur $ \mathcal{A} $ unter der Variableninterpretation $ \rho $ (in Zeichen: \red{$ \mathcal{A} \vDash_\rho \varphi $}) induktiv:
\begin{itemize}
    \item $ \mathcal{A} \vDash_\rho \perp $ gilt nicht
    \item $ \mathcal{A} \vDash_\rho R(t_1,\dots,t_k) $ falls $ (\rho(t_1),\dots, \rho(t_k)) \in R^\mathcal{A} $
    \item $ \mathcal{A} \vDash_\rho t_1 = t_2 $ falls $ \rho(t_1) = \rho(t_2) $ für $ t_1, t_2 \in T_\Sigma $
\end{itemize}
Für Formeln $ \varphi, \psi $ und $ x \in \texttt{Var} $:
\begin{itemize}
    \item $ \mathcal{A} \vDash_\rho \varphi \wedge \psi \textnormal{ falls } \mathcal{A}\vDash_\rho \varphi \textnormal{ und } \mathcal{A}\vDash_\rho \psi $
    \item $ \mathcal{A}\vDash_\rho \vee \psi $ falls $ \mathcal{A}\vDash_\rho \varphi $ oder $ \mathcal{A}\vDash_\rho \psi $
    \item $ \mathcal{A}\vDash_\rho \varphi \ra \psi $ falls nicht $ \mathcal{A}\vDash_\rho \varphi $ oder $ \mathcal{A}\vDash_\rho \psi $
    \item $ \mathcal{A}\vDash_\rho \lnot \varphi $ falls $ \mathcal{A}\vDash_\rho \varphi $ nicht gilt
    \item $ \mathcal{A}\vDash_\rho \exists x \varphi $ falls \red{???}
    \item $ \mathcal{A}\vDash_\rho \forall x \varphi $ falls \red{???}
\end{itemize}
Für $ x \in \texttt{Var} $ und $ a \in U_\mathcal{A}   $ sei \red{$ \rho[x \mapsto a]: \texttt{Var} \ra U_\mathcal{A} $} die Variableninterpretation, für die gilt:
\[
    (\rho[x \mapsto a])(y) = \begin{cases}
        \rho(y) & \textnormal{falls } x \neq y\\
        a & \textnormal{sonst}
    \end{cases}
\]
\begin{itemize}
    \item $ \mathcal{A}\vDash_\rho \exists x \varphi $ falls es $ a \in U_\mathcal{A} $ gibt mit $ \mathcal{A}\vDash_{\rho[x \mapsto a]} \varphi $
    \item $ \mathcal{A}\vDash_\rho \forall x \varphi $ falls $ \mathcal{A}\vDash_{\rho[x \mapsto a]} \varphi $ für alle $ a \in U_\mathcal{A} $
\end{itemize}


\subsection{\blue{Definition Modell}}
Sei $ \Sigma $ eine Signatur, $ \varphi $ eine Formel, $ \Delta $ eine Menge von Formeln und $ \mathcal{A} $ eine Struktur.\newline
\red{$ \mathcal{A} \vDash \varphi $} ($ \mathcal{A} $ ist \red{Modell} von $ \varphi $), falls $ \mathcal{A} \vDash_\rho \varphi $ für alle Variableninterpretation $ \rho $ gilt.\newline
\red{$ \mathcal{A} \vDash \Delta $}, falls  $ \mathcal{A} \vDash \psi $ für alle $ \psi \in \Delta $

\section{\blue{Definiton Erfüllbarkeit, Allgemeingültigkeit, Folgerung}}
Sei $ \Sigma $ eine Signatur, $ \varphi $ eine Formel, $ \Delta $ eine Menge von Formeln und $ \mathcal{A} $ eine Struktur.
\begin{itemize}
    \item $ \Delta $ ist \red{erfüllbar}, wenn es eine Struktur $ \mathcal{B} $ und Variableninterpretation $ \rho: \texttt{Var} \ra U_\mathcal{B} $ gibt mit $ \mathcal{B} \vDash_\rho \psi$ für alle $ \psi \in \Delta $.
    \item $ \varphi $ ist \red{semantische Folgerung} von $ \Delta $ (\red{$ \Delta \vDash \varphi $}) falls für alle Strukturen $ \mathcal{B} $ und alle Variableninterpretationen $ \rho $ gilt
    \[
        (\mathcal{B} \vDash_\rho \psi, \forall \psi \in \Delta) ~\RA~ \mathcal{B} \vDash_\rho \varphi
    \]
    \item $ \varphi $ ist \red{allgemeingültig}, falls $ \mathcal{B} \vDash_\rho \varphi $ für alle Strukturen $ \mathcal{B} $ und alle Variableninterpretationen $ \rho $ gilt
\end{itemize}
\paragraph{Bemerkung}
\[
    \varphi \textnormal{ allgemeingültig} ~\Leftrightarrow~ \emptyset \vDash \varphi ~\Leftrightarrow~ \{\lnot \varphi\} \textnormal{ nicht erfüllbar}
\]
Hierfür schreiben wir auch $ \vDash \varphi $.

\section{\blue{Substitutionen}}
Eine \red{Substitution} besteht aus einer Variable $ x \in \texttt{Var} $ und einem Term $ t \in T_\Sigma $, geschrieben \red{$ [x := t] $}.\ls
Die Formel $ \varphi[x:=t]  $ ist die Anwendung der Substitution $ [x:=t] $ auf die Formel $ \varphi $. Sie entsteht aus $ \varphi $, indem alle freien Vorkommen von $ x $ durch $ t $ ersetzt werden. \textit{Sie soll über $ t $ aussagen, was $ \varphi $ über $ x $ ausgesagt hat.}\ls
Dazu wird induktiv defininiert, was es heißt, die freuen Vorkommen von $ x $ im Term $ s \in T_\Sigma $ zu ersetzen:
\begin{itemize}
    \item $ x[x:=t] = t $
    \item $ y[x:=t] = t $ für $ y \in \texttt{Var}\setminus\{x\} $
    \item $ (f(t_1,\dots,t_k))[x:=t] = f(t_1[x:=t],\dots,t_k[x:=t]) $ für $ f \in \Omega $ mit $ \texttt{ar}(f) = k $ und $ t_1,\dots,t_k \in T_\Sigma $
\end{itemize}
\subsection{\green{Lemma}}
Seien $ \Sigma $ Signatur, $ \mathcal{A} $ Struktur, $ \rho $ Variableninterpretation, $ x \in \texttt{Var} $ und $ s,t \in T_\Sigma $. Dann gilt
\[
    \rho(s[x:=t]) = \rho[x \mapsto \rho(t)](s)
\]
\subsection{\blue{Definition Zulässigkeit}}
Sei $ [x:=t] $ Substitution und $ \varphi $ Formel.\newline
Die Substitution $ [x:=t] $ heißt \red{für $ \varphi $ zulässig}, wenn für alle $ y \in \texttt{Var} $ gilt
\[
    y \textnormal{ Variable in }t ~\RA~ \varphi \textnormal{ enthält weder } \exists y \textnormal{ noch } \forall y
\]
\subsection{\green{Lemma}}
Sei $ \Sigma $ Signatur, $ \mathcal{A} $ Struktur, $ \rho $ Variableninterpretation, $ x \in \texttt{Var} $ und $ t \in T_\Sigma $. Ist die Substitution $ [x:=t] $ für die Formel $ \varphi $ zulässig, so gilt
\[
    \mathcal{A} \vDash_\rho \varphi[x:=t] ~\Leftrightarrow~ \mathcal{A} \vDash_{\rho[x \mapsto \rho(t)]} \varphi
\]
\section{\blue{Natürliches Schließen}}
Die Regeln des natürlichen Schließens für aussagelogische Formeln sind auch anwendbar auf prädikatenlogische Formeln.
\subsection{\green{Reflexivität}}
Für jeden Term $ t $ ist
\begin{prooftree}
    \AxiomC{$  $}
    \UnaryInfC{$ t = t $}
\end{prooftree}
eine hypothesenlose Deduktion mit Konklusion $ t =t  $
\subsection{\green{Gleiches-Für-Gleiches}}
\paragraph{Idee:} ''Zunächst zeige ich, dass $ s $ die Eigenschaft $ \varphi $ hat. Dann zeige ich, dass $ s = t $. Also habe ich gezeigt, dass $ t $ die Eigenschaft $ \varphi $ hat.''
\begin{prooftree}
    \AxiomC{$ \varphi[x:=s] $}
    \AxiomC{$ s = t $}
    \BinaryInfC{$ \varphi[x:=t] $}
\end{prooftree}
\paragraph{Bedingung:} Über keine Variable aus $ s $ oder $ t $ wird in $ \varphi $ quantifiziert.
\subsection{\green{$ \forall $-Einführung/-Elimination}}
\paragraph{Einführung:}
\begin{prooftree}
    \AxiomC{$ \varphi $}
    \UnaryInfC{$ \forall x \varphi $}
\end{prooftree}
\paragraph{Bedingung:} $ x $ kommt in keiner Hypothese frei vor
\paragraph{Elimination:}
\begin{prooftree}
    \AxiomC{$ \forall x \varphi $}
    \UnaryInfC{$ \varphi[x := t] $}
\end{prooftree}
\paragraph{Bedingung:} Über keiner Variable aus $ t $ wird in $ \varphi $ quantifiziert
\subsection{\green{$ \exists $-Einführung/-Elimination}}
\paragraph{Einführung:}
\begin{prooftree}
    \AxiomC{$ \varphi[x:=t] $}
    \UnaryInfC{$ \exists x \varphi $}
\end{prooftree}
\paragraph{Bedingung:} Über keine Variable $ t $ wird in $ \varphi $ quantifiziert.
\paragraph{Elimination:}
\begin{prooftree}
    \AxiomC{$ \exists x \varphi $}
    \AxiomC{$ [\varphi] $}
    \noLine
    \UnaryInfC{$ \vdots $}
    \noLine
    \UnaryInfC{$ \sigma $}
    \BinaryInfC{$ \sigma $}
\end{prooftree}
\paragraph{Bedingung:} $ x $ kommt in den Hypothesen und in $ \sigma $ nicht frei vor

\section{\blue{Vollständigkeit}}
\subsection{\blue{Definition Konkretisierung}}
Eine Menge $ \Delta $ von Formeln hat \red{Konkretisierungen}, wenn für alle $ \exists x \varphi \in \Delta $ ein variablenloser Term $ t $ existiert mit $ \varphi [x := t] \in \Delta$
\subsection{\green{Satz}}
Sei $ \Delta $ eine maximal konsistente Menge von $ \Sigma $-Formeln. Dann existiert eine Signatur $ \Sigma^+ \supseteq \Sigma $ und eine maximal konsistente Menge von $ \Sigma^+ $-Formeln mit Konkretisierungen, so dass $ \Delta \subseteq \Delta^+ $.
\subsection{\green{Satz}}
Sei $ \Delta^+ $ maximal konsistente Menge von $ \Sigma^+ $-Formeln mit Konkretisierungen. Dann ist $ \Delta^+ $ erfüllbar.
\subsection{\green{Vollständigkeitssatz der Prädikatenlogik}}
Sei $ \Gamma $ eine Menge von Formeln und $ \varphi $ eine Formel. Dann gilt
\[
    \Gamma \vDash \varphi ~\RA~ \Gamma \vdash \varphi
\]
Insbesondere ist jede allgemeingültige Formel ein Theorem.
\paragraph{Bemerkung:} Dieser Satz ist im wesentlichen der Gödelsche Vollständigkeitssatz.
\subsection{\green{Satz}}
Sei $ \Gamma $ höchstens abzählbar unendliche und konsistente Menge von Formeln. Dann hat $ \Gamma $ ein höchsts abzählbar unendliches Modell.

\section{\blue{Vollständigkeit und Korrektheit}}
Seien $ \Gamma $ eine Menge von Formeln und $ \varphi $ eine Formel. Dann gilt
\[
    \Gamma \vdash \varphi ~\Leftrightarrow~ \Gamma \vDash \varphi
\]
Insbesondere ist eine Formel genau dann allgemeingültig, wenn sie ein Theorem ist.

\subsection{\green{Folgerung 1: Kompaktheit}}
Seien $ \Gamma $ eine u.U. unendliche Menge von Formeln und $ \varphi $ eine Formel mit $ \Gamma \vDash \varphi $. Dann existiert $ \Gamma' \subseteq \Gamma $ endlich mit $ \Gamma' \vDash \varphi $
\subsection{\green{Kompaktheitssatz}}
Sei $ \Gamma $ eine u.U unendliche Menge von Formeln. Dann gilt
\[
    \Gamma\textnormal{ erfüllbar} ~\Leftrightarrow~ \forall \Gamma' \subseteq \Gamma \textnormal{ endlich}: \Gamma' \textnormal{ erfüllbar}
\]
\subsection{\green{Satz}}
Sei $ \Delta $ eine u.U. unendliche Menge von Formeln, so dass für jedes $ n \in \mathbb{N} $ eine endliche Struktur $ \mathcal{A}_n $ mit $ \mathcal{A}_n \vDash \Delta $ existiert, die wenigstens $ n $ Elemente hat.\ls
Dann existiert eine unendliche Struktur $ \mathcal{A} $ mit $ \mathcal{A} \vDash \Delta $.
\subsection{\green{Folgerung 2: Löwenheim-Skolem}}
Sei $ \Gamma $ erfüllbare und höchstens abzählbar unendliche Menge von Formeln. Dann existiert ein höchstens abzählbar unendliches Modell von $ \Gamma $.
\subsection{\green{Folgerung 3: Semi-Entscheidbarkeit}}
Die Menge der allgemeingültigen Formeln ist semi-entscheidbar.

\subsection{\blue{Definition Horn-Formel}}
Eine \red{Horn-Formel} ist eine Konjuktion von Formeln der Form
\[
    \forall x_1 \forall x_2 \dots \forall x_n ((\lnot\perp \wedge \alpha_1 \wedge\dots \wedge \alpha_m) \ra \beta)
\]
wobei $ \alpha_1,\dots,\alpha_m  $ und $ \beta $ atomare Formeln sind.

\section{\blue{Definition Elementare Theorie}}
Sei $ \mathcal{A} $ eine Struktur. Dann ist $ \red{\texttt{Th}(\mathcal{A})} $ die Menge der prädikatenlogischen Formeln $ \varphi $ mit $ \mathcal{A} \vDash \varphi $. Diese Menge heißt die \red{(elementare Theorie von $ \mathcal{A} $)}.
\subsection{\green{Satz}}
Die Menge $ \texttt{Th}(\mathcal{N}) $ aller Sätze $ \varphi $ mit $ \mathcal{N} \vDash \varphi $ ist nicht entscheidbar.

\section{\red{Semi-Entscheidungsverfahren für allgemeingültige Formeln}}
\paragraph{Idee:}
\begin{itemize}
    \item Berechne aus Formel $ \psi $ eine Menge $ E $ von aussagelogischen Formeln mit
    \[
        E \textnormal{ unerfüllbar }  ~\RA~ \lnot \psi \textnormal{ unerfüllbar } ~\RA~ \psi \textnormal{ allgemeingültig}
    \]
    \item Suche endliche unerfüllbare Teilmenge $ E' $ von $ E $
\end{itemize}
\subsection{\blue{Definition Erfüllbarkeitsäquivalenz}}
Zwei Formeln $ \varphi $ und $ \psi $ heißen \red{erfüllbarkeitsäquivalent},wenn gilt:
\[
    \varphi \textnormal{ erfüllbar } ~\RA~ \psi \textnormal{ erfüllbar}
\]
\subsection{\blue{Definition Gleichungsfreiheit}}
Eine Formel ist \red{gleichungsfrei}, wenn sie keine Teilformel der Form $ s = t $ enthält.
\paragraph{Notation:}
\begin{itemize}
    \item Sei $ \Sigma = (\Omega, \texttt{Rel}, \texttt{ar}) $ endliche Signatur und $ \varphi $ Formel
    \item \red{$ \Sigma_{\texttt{Gl}} $}$ = (\Omega, \texttt{Rel} \uplus \{\texttt{Gl}\}, \texttt{ar}_{\texttt{Gl}}) $ mit $ \texttt{ar}_{\texttt{Gl}}(f) = \texttt{ar}(f)  $ für alle $ f \in \Omega \cup \texttt{Rel} $ und $ \texttt{ar}_{\texttt{Gl}}(\texttt{Gl}) = 2 $
    \item Für eine Formel $ \varphi $ bezeichnet $ \varphi_{\texttt{Gl}} $ die $ \Sigma_{\texttt{Gl}} $-Formel, die aus $ \varphi $ entsteht, indem alle Vorkommen von Teilformeln $ s = t $ durch $ \texttt{Gl}(s,t) $ ersetzt werden
\end{itemize}
\section{\blue{Kongruenz und Äquivlenz}}

\subsection{\blue{Definition Kongruenz}}
Sei $ \mathcal{A} $ eine Struktur und $ \sim  $ eine binäre Relation auf $ U_\mathcal{A} $. Die Relation $ \sim $ heißt \red{Kongruenz auf $ \mathcal{A} $}, wenn gilt:
\begin{itemize}
    \item $ \sim $ ist eine Äquivalenzrelation (reflexiv, transitiv, symmetrisch)
    \item fü alle $ f \in \Omega $ mit $ k = \texttt{ar}(f) $ und alle $ a_1, b_1,\dots,a_k,b_k \in U_\mathcal{A} $ gilt
    \[
        a_1 \sim b_1,~ a_2 \sim b_2,~ \dots ,~ a_k \sim b_k ~\RA~ f^\mathcal{A}(a_1,\dots,a_k) \sim f^\mathcal{A}(b_1,\dots,b_k)
    \]
    \item für alle $ R \in \texttt{Rel} $ mit $ k = \texttt{ar}(R) $ und alle $ a_1,b_1,\dots, a_k, b_k \in U_\mathcal{A}  $ gilt
    \[
        a_1 \sim b_1,~\dots,~a_k \sim b_k,~ (a_1,\dots,a_k) \in R^\mathcal{A} ~\RA~ (b_1,\dots,b_k) \in R^\mathcal{A}
    \]
\end{itemize}

\subsection{\blue{Definition Äquivalenzklasse}}
Sei $ \mathcal{A} $ eine Struktur und $ \sim $ eine Kongruenz auf $ \mathcal{A} $.
\begin{itemize}
    \item Für $ a \in U_\mathcal{A} $ sei \red{$ [a] $}$ = \{ b \in U_\mathcal{A} \mid a \sim b \} $ die \red{Äquivalenzklasse} von $ a $ bzgl. $ \sim $
    \item Dann definieren wir den \red{Quotienten $ \mathcal{B} = \mathcal{A}/\sim  $} von $ \mathcal{A} $ bzgl. $ \sim $ wie folgt
    \item[] \begin{itemize}
        \item $ U_\mathcal{B} = U_\mathcal{A}/\sim = \{[a] \mid a \in U_\mathcal{A}\}  $
        \item Für jedes $ f \in \Omega $ mit $ \texttt{ar}(f) = k $ und alle $  a_1,\dots,a_k \in U_\mathcal{A} $ setzen wir
        \[
            f^\mathcal{B}([a_1],\dots,[a_k]) = [f^\mathcal{A}(a_1,\dots,a_k)]
        \]
        \item Für jedes $ R \in \texttt{Rel} $ mit $ \texttt{ar}(R) = k $ setzen wir
        \[
            R^\mathcal{B} = \{([a_1],\dots,[a_k]) \mid (a_1,\dots,a_k) \in R^\mathcal{A}\}
        \]
    \end{itemize}
    \item Sei $ \rho: \texttt{Var} \ra U_\mathcal{A} $ Variableninterpretation. Dann definiere die Variableninterpretation
    \[
        \red{\rho/\sim}: \texttt{Var} \ra U_\mathcal{B}: x \mapsto [\rho(x)]
    \]
\end{itemize}
\subsection{\green{Lemma 1}}
Sei $ \mathcal{A} $ Struktur, $ \rho: \texttt{Var} \ra U_\mathcal{A} $ Variableninterpretation und $ \sim $ Kongruenz. Seien weiter $ \mathcal{B} = \mathcal{A}/\sim $ und $ \rho_\mathcal{B} = \rho/\sim $. Dann gilt für jeden Term $ t $:
\[
    [\rho(t)] = \rho_\mathcal{B}(t)
\]
\subsection{\green{Lemma 2}}
Sei $ \mathcal{A} $ Struktur, $ \sim $ Kongruenz und $ \mathcal{B} = A/\sim $. Dann gilt für alle $ R \in \texttt{Rel} $ mit $ k = \texttt{ar}(R) $ und alle $ c_1,\dots,c_k \in U_\mathcal{A} $:
\[
    ([c_1], \dots, [c_k]) \in R^\mathcal{B} ~\Leftrightarrow~ (c_1,\dots,c_k) \in R^\mathcal{A}
\]
\subsection{\green{Satz}}
Seien $ \mathcal{A} $ $ \Sigma_\texttt{Gl} $-Struktur und $ \rho: \texttt{Var} \ra U_\mathcal{A} $ Variableninterpretation, so dass $ \sim = \texttt{Gl}^\mathcal{A} $ Kongruenz auf $ \mathcal{A}  $ ist.\ls
Seien $ \mathcal{B} = A/\sim $ und $ \rho_\mathcal{B} = \rho/\sim $. Dann gilt für alle Formeln $ \varphi $:
\[
    A \vDash_\rho \varphi_\texttt{Gl} ~\RA~ \mathcal{B} \vDash_{\rho_\mathcal{B}} \varphi
\]
\subsection{\green{Lemma}}
Aus einer endlichen Signatur $ \Sigma $ kann ein gleichungsfreier Horn-Satz $ \texttt{Kong}_\Sigma $ über $ \Sigma_\texttt{Gl} $ berechnet werden, so dass fü® alle $ \Sigma_\texttt{Gl} $-Strukturen $ \mathcal{A} $ gilt:
\[
    \mathcal{A} \vDash \texttt{Kong}_\Sigma ~\RA~ \texttt{Gl}^\mathcal{A} \textnormal{ ist eine Kongruenz}
\]
\subsection{\green{Satz}}
Aus einer endlichen Signatur $ \Sigma $ und einer Formel $ \varphi $ kann eine gleichungsfreie und erfüllbarkeitsäquivalente $ \Sigma_\texttt{Gl} $-Formel $ \varphi' $ berechnet werden.\ls
Ist $ \varphi $ Horn-Formel, so ist auch $ \varphi' $ Horn-Formel.

\section{\blue{Skolemform}}
\paragraph{Ziel:} Jede $ \Sigma $-Formel $ \varphi $ ist erfüllbarkeitsäquivalent zu einer $ \Sigma' $-Formel
\[
    \varphi' = \forall x_1 \forall x_2 \dots \forall x_k \psi
\]
wobei $ \psi $ keine Quantoren enthält, $ \varphi' $ heißt \red{in Skolemform}.
\paragraph{Vorgehen:}
\begin{itemize}
    \item[(1)] Quantoren nach vorne (Pränexform)
    \item[(2)] Existenzquantoren eliminieren
\end{itemize}
\paragraph{Bemerkung:} Es gibt Formeln ohne äquivlente Formel in Skolemform.

\subsection{\blue{Äquivalenz von Formeln}}
Zwei Formeln $ \varphi $ und $ \psi $ sind \red{äquivalent} (kurz $ \varphi \red{\equiv} \psi $), wenn für alle Strukturen $ A $ und alle Variableninterpretationen $ \rho $ gilt
\[
    \mathcal{A} \vDash_\rho \varphi ~\RA~ \mathcal{A} \vDash_\rho \psi
\]
\subsection{\green{Lemma}}
Seien $ \Box \in \{\exists, \forall\} $ und $ \oplus \in \{\wedge, \vee, \ra, \leftarrow\} $. \newline
Sei $ \varphi = (\Box x \alpha) \oplus \beta $ und sei $ y $ eine Variable, die weder in $ \alpha $ noch in $ \beta $ vorkommt. Dann gilt
\[
    \varphi \equiv \begin{cases}
        \Box y (\alpha[x:=y] \oplus \beta) & \textnormal{ falls } \oplus \in \{\wedge, \vee, \Leftarrow\}\\
        \forall y (\alpha[x:=y] \ra \beta) & \textnormal{ falls } \oplus = \ra,~ \Box = \exists\\
        \exists y (\alpha[x := y] \ra \beta) & \textnormal{ falls } \oplus = \ra,~ \Box = \forall
    \end{cases}
\]
\subsection{\green{Satz}}
Aus einer endlichen Signatur $ \Sigma $ und einer Formel $ \varphi $ kann eine äquivalente Formel
\[
    \varphi' = \Box_1x_1\dots\Box_kx_k \psi
\]
berechnet werden. Eine Formel $ \varphi' $ dieser Form heißt \red{Pränexform}.\ls
Ist $ \varphi $ gleichungsfrei, so ist auch $ \varphi' $ gleichungsfrei.

\subsection{\red{Existenzquantoren eliminieren}}
Sei $ \varphi = \forall x_1 \forall x_2\dots\forall x_m \exists y \psi $ Formel in Pränexform. Sei $ g \notin \Omega $ ein neues $ m $-stelliges Funktionssymbol.\ls
Setze \red{$ \varphi' $}$ = \forall x_1 \forall x_2\dots \forall x_m \psi [y := g(x_1,\dots,x_m)] $.\ls
Offensichtlich hat $ \varphi' $ einen Existenzquantor weniger als $ \varphi $. $ \varphi' $ ist außerdem keine $ \Sigma $-Formel, da $ g \notin \Omega $, sondern eine Formel über einer erweiterten Signatur.

\subsection{\green{Lemma}}
Die Formeln $ \varphi $ und $ \varphi' $ sind erfüllbarkeitsäquivalent.
\subsection{\green{Satz}}
Aus einer Formel $ \varphi $ kann man eine erfüllbarkeitsäquivalente Formel $ \bar{\varphi}  $ in Skolemform berechnen. \newline
Ist $ \varphi $ gleichungsfrei, so auch $ \bar \varphi $.

\section{\blue{Herbrand-Strukturen und Modelle}}
Sei $ \Sigma = (\Omega, \texttt{Rel}, \texttt{ar})$ eine Signatur. Wir nehmen im folgenden an, dass $ \Omega $ mindestens ein Konstantensymbol enthält. \ls
Das \red{Herbrand-Universum $  D(\Sigma) $} ist die Menge aller variablenfreien $ \Sigma $-Terme. \ls
\paragraph{Beispiel:}
\begin{align*}
    & \Omega = \{b, f\} \qquad \texttt{ar}(b) = 0 \qquad \texttt{ar}(f) = 1\\
    & D(\Sigma) = \{ b, f(b), f(f(b)), f(f(f(b))),\dots \}
\end{align*}
Eine $ \Sigma $-Struktur $ \mathcal{A} = (U_\mathcal{A}, (f^\mathcal{A})_{f \in \Omega}, (R^\mathcal{A})_{R \in \texttt{Rel}}) $ ist eine \red{Herbrand-Struktur}, wenn folgendes gilt
\begin{itemize}
    \item $ U_\mathcal{A} = D(\Sigma)  $
    \item für alle $ f \in \Omega $ mit $ \texttt{ar}(f) = k $ und alle $ t_1,\dots,t_k \in D(\Sigma) $ ist
    \[
        f^\mathcal{A}(t_1,\dots,t_k) = f(t_1,\dots,t_k)
    \]
\end{itemize}
Für jede Herbrand-Struktur $ \mathcal{A} $, alle Variableninterpretationen $ \rho $ und alle variablenfreien Terme $ t $ gilt dann $ \rho(t) = t $.\ls
Ein \red{Herbrand-Modell} von $ \varphi $ ist eine \red{Herbrand-Struktur}, die gleichzeitig ein Modell von $ \varphi $ ist.

\subsection{\green{Satz}}
Sei $ \varphi $ eine gleichungsfreie Aussage in Skolemform. $ \varphi $ ist genau dann erfüllbar, wenn $ \varphi $ ein Herbrand-Modell besitzt.

\subsection{\blue{Herband-Expansion}}
\paragraph{Frage:} Wie erkennt man, ob eine gleichungsfreie Aussage in Skolemform ein Herband-Modell hat?\ls
Die \red{Herbrand-Expansion} von $ \varphi $ ist die Menge der Aussagen
\[
    E(\varphi) = \{ \psi[y_1 := t_1]\dots[y_n := t_n] \mid t_1,\dots,t_n \in D(\Sigma) \}
\]
Die Formeln von $ E(\varphi) $ entstehen also aus $ \psi $, in dem die (variablenfreien) Terme aus $ D(\Sigma) $ in jeder möglichen Weise in $ \psi $ substituiert werden. \ls
Wir betrachen die Herbrand-Expansion von $ \varphi $ als eine Menge von \red{aussagelogischen Formeln}. Die atomaren Formeln sind hierbei von der Gestallt $ (P(t_1,\dots,t_k) $ für $ P \in \texttt{Rel} $ mit $ \texttt{ar}(P) = k $ und $ t_1,\dots,t_k \in D(\Sigma) $.

\subsection{\red{Konstruktion}}
Sei $ \mathcal{B}: \{P(t_1,\dots,t_k) \mid P \in \texttt{Rel}, k = \texttt{ar}(P), t_1,\dots,t_k \in D(\Sigma)\} \ra \mathbb{B} $ eine $ \mathbb{B}-Belegung $.
Die hiervon \red{induzierte Herbrand-Struktur $ \mathcal{A}_\mathcal{B} $} ist gegeben durch
\[
    P^{\mathcal{A}_\mathcal{B}} = \{ (t_1,\dots,t_k) \mid t_1,\dots,t_k \in D(\Sigma), \mathcal{B}(P(t_1,\dots,t_k)) = 1 \}
\]
für alle $ P \in \texttt{Rel} $ mit $ \texttt{ar}(P) =k$
\subsection{\green{Lemma}}
Für jede quantoren- und gleichungsfreie Aussage $ \alpha $ und jede Variableninterpretation $ \rho $ in $ \mathcal{A}_\mathcal{B} $ gilt
\[
    \mathcal{A}_\mathcal{B} \vDash_\rho \alpha ~\RA~ \mathcal{B}(\alpha) = 1
\]
\subsection{\green{Lemma}}
Sei $ \varphi = \forall y_1 \forall y_2 \dots \forall y_n \psi $ gleichungsfreie Aussage in Skolemform. Sie hat genau dann ein Herbrand-Modell, wenn die Formelmenge $ E(\varphi) $ (im aussagelogischen Sinn) erfüllbar ist.

\subsection{\green{Satz von Gödel-Herbrand-Skolem}}
Sei $ \varphi $ gleichungsfreie Aussage in Skolemform. Die ist genau dann erfüllbar, wenn die Formelmenge $ E(\varphi) $ (im aussagelogischen Sinn) erfüllbar ist.

\subsection{\green{Satz von Herbrand}}
Eine gleichungsfreie Aussage $ \varphi $ in Skolemform ist genau dann unerfüllbar, wenn es eine endliche Teilmenge von $ E(\varphi) $ gibt, die (im aussagelogischen Sinn) unerfüllbar ist.

\subsection{\red{Algorithmus von Gilmore}}
Sei $ \varphi $ gleichungsfreie Aussage in Skolemform un sei $ \alpha_1,\alpha_2,\dots $ eine Aufzählung von $ E(\varphi) $.
\paragraph{Algorithmus von Gilmore}
\begin{itemize}
    \item[] \textbf{Eingabe:} $ \varphi $
    \item[] $ n:=0 $
    \item[] \texttt{repeat} $ n:= n+1 $
    \item[] \texttt{until} $ \{ \alpha_1,\alpha_2,\dots,\alpha_n \} $ ist unerfüllbar
    \item[] \texttt{return} ''unerfüllbar''
\end{itemize}
\paragraph{Bemerkung:} ''$ \{ \alpha_1,\alpha_2,\dots,\alpha_n \} $ ist unerfüllbar'' ist nachzuweisen mit den Mitteln der Aussagenlogik, z.B. Wahrheitswertetabelle.

\paragraph{Folgerung}
Sei $ \varphi $ eine gleichungsfreie Aussage in Skolemform. Dann gilt:
\begin{itemize}
    \item Wenn die Eingabeformel $ \varphi $ unerfüllbar ist, dann terminiert der Algorithmus und gibt ''unerfüllbar'' aus.
    \item Wenn die Eingabeformel $ \varphi $ erfüllbar ist, dann terminiert der Algorithmus nicht.
\end{itemize}
\subsection{\green{Bestimmung von Lösungen}}
\paragraph{Folgerung 1:} Sei $ \varphi = \bigwedge_{1 \leq i \leq n} \varphi_i $ gleichungsfreie Horn-Formel der Prädikatenlogik. Dann ist $ \varphi $ genau dann unerfüllbar, wenn $ \bigcup_{1 \leq i \leq n} E(\varphi_i)$ im aussagelogischen Sinne unerfüllbar ist.
\paragraph{Folgerung 2:} Eine gleichungsfreie Horn-Formel der Prädikatenlogik $ \varphi = \bigwedge_{1 \leq i \leq n} \varphi_i $ ist genau dann unerfüllbar, wenn es eine SLD-Resolution
\[
    (M_0 \ra \perp, M_1 \ra \perp,\dots, M_m \ra \perp)
\]
aus $ \bigcup_{1 \leq i \leq n}E(\varphi_i) $ mit $ M_m = \emptyset $ gibt.

\section{\blue{Substitutionen}}
Eine \red{verallgemeinerte Substitution} $ \sigma $ ist eine Abbildung der Menge der Variablen in die Menge aller Terme, so dass nur endlich viele Variablen $ x $ existeren mit $ \sigma(x) \neq x$.\ls
Sei \red{$ \texttt{Def}(\sigma) $}$ = \{ x \textnormal{ Variable } \mid x \neq \sigma(x) \} $ der \red{Definitionsbereich} der verallgemeinerten Substitution $ \sigma $.
Für einen Term $ t $ definieren wir den Term $ \red{t \sigma} $ (Anwendung der verallgemeinerten Substitution $ \sigma $ auf den Term $ t $) wie folgt induktiv
\begin{itemize}
    \item $ x \sigma = \sigma(x) $
    \item $ [f(t_1,\dots,t_k)]\sigma = f(t_1\sigma,\dots,t_k \sigma $ für Terme $ t_1,\dots,t_k, f \in \Omega $
\end{itemize}
Für eine atomare Formel $ \alpha = P(t_1,\dots, t_k) $ (d.h. $ P \in \texttt{Rel}, \texttt{ar}(P) =k, t_1,\dots,t_k $ Terme ) sei
\[
    \red{\alpha \sigma} = P(t_1 \sigma,\dots, t_k \sigma)
\]
\subsection{\blue{Verknüpfung von Substitutionen}}
Sind $ \sigma_1, \sigma_2 $ verallgemeinerte Substitutionen, so definieren wir eine neue verallgemeinerte Substitution \red{$ \sigma_1 \sigma_2 $} durch
\[
    (\sigma_1 \sigma_2)(x) = (x\sigma_1)\sigma_2
\]
\subsection{\green{Lemma}}
Seien $ \sigma $ Substitution, $ x $ Variable und $ t $ Term, so dass
\begin{itemize}
    \item $ x \notin \texttt{Def}(\sigma) $
    \item $ x $ kommt in keinem der Terme $ y \sigma $ mit $ y \in \texttt{Def}(\sigma) $ vor
\end{itemize}
Dann gilt
\[
    [x:=t]\sigma = \sigma[x:=t\sigma]
\]
\subsection{\blue{Unifikator}}
Gegeben seien zwei gleichungsfreie Atomformeln $ \alpha, \beta $. Eine Substitution $ \sigma $ heißt \red{Unifikator} von $ \alpha $ und $ \beta $, falls
\[
    \alpha\sigma = \beta\sigma
\]
Ein Unifikator $ \sigma $ heißt \red{allgemeinster Unifikator}, falls für jeden Unifikator $ \sigma' $ eine Substitution $ \tau $ mit $ \sigma' = \sigma \tau $ existiert.

\subsection{\blue{Variablenumbenennung}}
Eine \red{Variablenumbenennung} ist eine Substitution $ \rho $, die $ \texttt{Def}(\rho) $ injektiv in die Menge der Variablen abbildet.
\subsection{\green{Lemma}}
Sind $ \sigma_1, \sigma_2 $ allgemeinste Unifikatoren von $ \alpha, \beta $, so existiert eine Variablenumbenennung $ \rho $ mit $\sigma_2 = \sigma_1 \rho$.

\subsection{\red{Unifikationsalgorithmus}}
\begin{itemize}
    \item[\textbf{Eingabe: }] Paar $ (\alpha,\beta) $ gleichungsfreier Atomformeln
    \item[] $ \sigma := $ Substitution mit $ \texttt{Def}(\sigma) = \emptyset$ (d.h. Identität)
    \item[\textbf{while }] $ \alpha \sigma \neq \beta \sigma $ \textbf{do}
    \item[] \begin{itemize}
        \item[] Suche die erste Position, an der sich $ \alpha\sigma $ und $ \beta\sigma $ unterscheiden
        \item[\textbf{if }] keines der beiden Symbole an dieser Position ist eine Variable
        \item[\textbf{then }] stoppe mit ''nicht unifizierbar''
        \item[\textbf{else }] sei $ x $ die Variable und $ t $ der term in der anderen Atomformel
        \item[] \begin{itemize}
            \item[\textbf{if }] $ x $ kommt in $ t $ vor
            \item[\textbf{then }] stoppe mit ''nicht unifizierbar''
            \item[\textbf{else }] $ \sigma := \sigma[x:=t] $
        \end{itemize}
    \end{itemize}
    \item[\textbf{endwhile}]
    \item[\textbf{Ausgabe: }]$ \sigma $
\end{itemize}
\subsection{\green{Satz}}
\begin{itemize}
    \item Der Unifikationsalgorithmus terminiert für jede Eingabe
    \item Wenn die Eingabe nicht unifizierbar ist, so terminiert der Algorithmus mit der Ausgabe ''nicht unifizierbar''
    \item wenn die Eingabe unifizierbar ist, dann findet der Algorithmus den allgemeinsten Unifikator
\end{itemize}

\section{\red{Prädikatenlogische SLD-Resolution}}
Sei $ \Gamma $ eine Menge von gleichungsfreien Horn-Klauseln der Prädikatenlogik. Eine \red{SLD-Resolution aus $ \Gamma $} ist eine Folge
\[
    ((M_0 \ra \perp, \sigma_0), (M_1 \ra \perp, \sigma_1),\dots, (M_m \ra \perp, \sigma_m))
\]
von Horn-Klauseln und Substitutionen mit
\begin{itemize}
    \item $ (M_0 \ra \perp) \in \Gamma $ und $ \texttt{Def}(\sigma_0) = \emptyset $
    \item für alle $ 0 \leq n < m $ existieren $ \emptyset \neq Q \subseteq M_n,~ (N \ra \alpha) \in \Gamma $ und Variablenumbenennung $ \rho $, so dass
    \item \begin{itemize}
        \item $ (N \cup \{\alpha\}) \rho $ und $ M_n $ variablendisjunkt
        \item $ \sigma_{n+1} \textnormal{ ein allgemeinster Unifikator von } \alpha \rho $ und $ Q $
        \item $ M_{n+1} = (M_n \setminus Q \cup N \rho) \sigma_{n+1} $
    \end{itemize}
\end{itemize}
\subsection{\green{Lemma}}
Sei $ \Gamma $ die Menge von gleichungsfreien Horn-Klauseln der Prädikatenlogik und $ ((M_n \ra \perp, \sigma_n))_{0 \leq n \leq m} $ eine SLD-Resolution aus $ \Gamma \cup \{M_0 \ra \perp\} $ mit $ M_m = \emptyset $.\newline
Dann gilt $ \Gamma \vDash \psi \sigma_0\sigma_1\sigma_2\dots\sigma_m $ für alle $ \psi \in M_0 $.
\subsection{\green{Lemma}}
Sei $ \Gamma $ eine Menge von definiten gleichungsfreien Horn-Klauseln der Prädikatenlogik, sei $ M \ra \perp $ eine gleichungsfreie Horn-Klausel und sei $ \upsilon $ Substitution, so dass $ M \upsilon $ variablenlos ist und $ \Gamma \vDash M \upsilon $ gilt.\newline
Dann existiert eine prädikatenlogische SLD-Resolution
\[
    ((M_n \ra \perp, \sigma_n))_{0 \leq n \leq m} \textnormal{ aus } \Gamma \cup \{M \ra \perp\}
\]
und eine Substitution $ \tau $ mit
\[
    M_0 = M, M_m = \emptyset \textnormal{ und } M_0\sigma_0\sigma_1\dots\sigma_m \tau = M\upsilon
\]
\subsection{\green{Satz}}
Sei $ \Gamma $ eine Menge von definiten gleichungsfreien Horn-Klauseln der Prädikatenlogik, sei $ M \ra \perp $ eine gleichungsfreie Horn-Klausel und sei $ \upsilon $ Substitution, so dass $ M \upsilon $ variablenlos ist. Dann sind äquivlent:
\begin{itemize}
    \item $ \Gamma \vDash M \upsilon $
    \item Es existiert eine SLD-Resolution $ ((M_n \ra \perp, \sigma_n))_{0 \leq n \leq m} \textnormal{ aus } \Gamma \cup \{M \upsilon \ra \perp\} $ und eine Substitution $ \tau $ mit $ M_0 = M, M_m = \emptyset $ und $ M_0\sigma_0\sigma_1\dots\sigma_m \tau = M\upsilon $
\end{itemize}

\end{document}